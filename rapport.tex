\documentclass[a4paper, titlepage]{article}
\usepackage{xunicode}
\usepackage{fontspec}
\usepackage[frenchb]{babel}
\usepackage{graphicx}
\renewcommand{\familydefault}{\sfdefault}

\renewcommand{\nobreakspace}{\nobreak\ }

\title{Projet HADLf: Conception et réalisation}
\author{Julien Durillon \and Alexandre Garnier}
\date{\today}

\begin{document}

	\maketitle

	\tableofcontents\clearpage

	\section{Niveau M2}
		\subsection{Conception}
			Ce niveau décrit l'architecture d'un système à composants.
		
			%Diagramme général M2
			\begin{figure}[ht]
				\centering
				\includegraphics[width=1.00\textwidth]{M2.png}
				\caption{M2: Modèle de l'architecture à composant}
				\label{fig:m2}
			\end{figure}
		
		
			La figure \ref{fig:m2} exprime les différents éléments de l'architecture à
			composants et leurs liens.
		
			\paragraph{Composant}
			
				Cette classe est abstraite et doit permettre à n'importe quelle
				classe de se déclarer comme composant en l'étendant.
				
			\paragraph{SimpleComposant et Configuration}
			
				Ces deux classes --- avec Composant--- permettent de fournir un
				pattern composite, demandé par la spécification. Configuration
				est l'objet composite.
			
			\paragraph{Interface}
			
				Un Composant possède des interfaces, qui sont de deux types: Port et
				Service. Ces interfaces sont soit fournies soit requises. Voir page
				\pageref{def:reqfour} pour les interfaces fournies et requises.
				
				
			\paragraph{Connecteur}
			
				Un connecteur connecte une interface fournie (roleFrom) d'un
				composant, et une interface requise (roleTo) d'un autre. Les
				interfaces doivent être de différents composants.
				
				Les connecteurs sont définis dans les configurations. Ils ne
				connectent que les interfaces des composants contenus par la
				configuration.
				
			\paragraph{Binding}
			
				Un Binding fonctionne comme un connecteur, à la différence qu'il
				connecte une interface d'une configuration à une interface d'un de
				ses composants. Les deux interfaces connectées doivent être toutes
				les deux fournies ou requises.
				
				L'appel d'une interface requise d'une configuration doit être passé
				à l'interface requise du composant connecté.
				
				La mise à disposition d'une interface fournie d'un composant doit
				donner lieu à la mise à disposition de l'interface fournie de sa
				configuration associée.
				
			\paragraph{Requise vs fournie}
				\label{def:reqfour}
				Une interface requise est une interface par laquelle on va pouvoir
				passer des messages à un composant. Son nom vient du fait qu'elle va
				requérir des informations.
				
				Une interface requise est une interface qui va fournir des
				informations.
				
				Quand une interface requise est connectée à une interface fournie,
				le connecteur se déclenchera quand l'interface fournies se déclarera
				prête à envoyer des informations.
				
			
			
		\subsection{Implémentation}
		
	
	\section{Conception et implémentation du M1 : système client-serveur}
	
	
	
	\section{Modélisation de l'architecture à composant}
	
	
	
	
\end{document}
